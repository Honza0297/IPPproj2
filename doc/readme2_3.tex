\documentclass[a4paper, 11pt]{article}

\usepackage[czech]{babel}
\usepackage{times}
\usepackage[text={17cm,24cm}, top=3cm, left=2cm]{geometry}
\usepackage[utf8]{inputenc}
\setlength{\headheight}{20.0mm}
\usepackage{fancyhdr}
\pagestyle{fancy}
\usepackage{graphics}
\usepackage {array}
\usepackage{pdflscape}
\usepackage[czech, ruled, vlined, linesnumbered, longend, noline]{algorithm2e}
\usepackage{multirow}

\begin{document}
\catcode`\-=12 %Mělo by to vyřešit problém s cline... jestli ne, tak už nevím
%Uvodni strana
\pagestyle{fancy}
\fancyhf{}
\fancyhead[R]{\scalebox{0.5}{\includegraphics{logo.png}}}
\fancyhead[L]{\Large Jan Beran\\ \texttt{xberan43}\\ \today}

\setlength{\parindent}{0mm}
\large{\textbf{Implementační dokumentace ke 2. a 3. úloze do IPP 2018/2019}}\\\
\large{\textbf{Jméno a příjmení: Jan Beran}}\\\
\large{\textbf{Login: xberan43}}\\\

\large{\textbf{Úvod}}\\\
Tato dokumentace stručně popisuje skripty interpret.py a test.php.\\

\large{{\textbf{Zadání}}\\
Zadáním bylo vytvořit dva skripty; interpret jazyka IPPcode19 převedeného do XML a testovací rámec jak pro interpret, tak pro parser, který byl součástí prvního odevzdání. Pro interpret byl specifikovaným jazykem Python 3.6, pro testovací rámec PHP7.3.\\

\large{{\textbf{Implementace interpretu}}\\
Interpret je implementován s využitím objektově orientovaných principů, které výrazně usnadnily celou tvorbu skriptu. V mém návrhu je hlavním objektem objekt interpret, starající se o samostatnou interpretaci kódu. Rámce pro proměnné a jejich zásobník byly též implementovány jako samostatné objekty.\\
Interpret na vstupu přijímá XML podobu IPPcode19 a vstupní hodnoty pro interpretací kódu a na výstupu vypíše výstup interpretovanéh kódu, případně chybové hlášení, pokud se interpretovaný kód nebo sám interpret dostane do chybového stavu.\\


\large{{\textbf{Spouštění interpretu}}\\
Skript se spouští pomocí příkazové řádky jako skript v jazyce Python3. Pracuje s několika parametry:\\
\begin{itemize}
	\item -{}-help: Vypsání krátké nápovědy k programu
	\item -{}-source=\textit{source}: specifikuje soubor pro XML reprezentaci IPPcode19, ze kterého mají být načítána data
	\item -{}input=\textit{input}: specifikuje soubor se vstupy pro interpretovaný kód
\end{itemize}
Vždy alespoň jeden z dvojice posledně zmíněných parametrů musí být zadán. Pokud jeden z parametrů chybí, příslušná data se načítají ze standardního vstupu.\\

\large{{\textbf{Implementace testovacího rámce}}\\
Testovací rámec je implementován funkcionálně s důrazem na znovupoužitelnost pro další projekty, jelikož tento skript plánují využívat i pro testování dalších projektů, bude-li to vhodné. Kromě serveru Merlin, který jsem považoval za referenční stroj, skript funguje (a byl testován) i pod OS Windows, a to z výše zmíněného důvodu. \\
\newpage
\large{{\textbf{Spouštění testovacího rámce}}\\
Aplikace se spouští pomocí příkazové řádky jako skript pro PHP7.3. Pracuje s několika parametry:\\
\begin{itemize}
	\item -{}-help: Vypsání krátké nápovědy k programu
	\item -{}-recursive: Zda mají být při hledání testů prohledávány i podsložky
	\item -{}-directory=\textit{path}: Určení složky, ve které se nacházejí testy. Při absenci tohoto přepínače se testy hledají v aktuálním umístění.
	\item -{}-parse-script=\textit{file}: Určení cesty k analyzátoru zdrojového kódu parse.php. Při absenci přepínače bude skript hledán v aktuálním adresáři.
	\item -{}-int-script=\textit{file}:Určení cesty k interpretu zdrojového kódu interpret.py. Při absenci přepínače bude skript hledán v aktuálním adresáři.
	\item -{}-parse-only: Při tomto přepínači se bude testovat pouze skript pro analýzu zdrojovéh kódu. Nesmí se kombinovat s přepínači -{}-int-script nebo -{}-int-only.
	\item -{}-int-only: Při tomto přepínači se bude testovat pouze skript pro interpretaci zdrojovéh kódu. Nesmí se kombinovat s přepínači -{}-parse-script nebo -{}-parse-only.
\end{itemize}
\large{{\textbf{Závěr}}\\
Skripty byly implementovány samostatně a odpovídají všem náležitostem zadání.\\
\end{document}





































